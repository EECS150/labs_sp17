\documentclass[11pt]{article}

\usepackage{float}
\usepackage{hyperref}
\usepackage{fullpage}
\usepackage{verbatim}
\usepackage{moreverb}
\usepackage{graphicx}
\usepackage{array}
\usepackage{booktabs}
\usepackage{parskip}
\usepackage{tikz}
\usetikzlibrary{arrows,automata, positioning}
\usepackage{minted}
\let\verbatiminput=\verbatimtabinput
\def\verbatimtabsize{4\relax}
\graphicspath{{images/}}

\begin{document}
\title{EECS 151/251A FPGA Lab\\
Lab 6: Multi-bit Clock-Crossing, FIFOs, UART Piano, Project Intro}

\author{Prof. Elad Alon \\
TAs: Vighnesh Iyer, Bob Zhou \\Department of Electrical Engineering and Computer Sciences\\
College of Engineering, University of California, Berkeley}
\date{}
\maketitle

\tableofcontents

\section{Intro}
In this lab, you will integrate the components you created in labs 4 and 5 (UART and AC97 controller). You will begin by building 2 varieties of FIFOs, asynchronous and synchronous, and verifying their functionality using a block-level testbench. You will then modify your AC97 controller to use a FIFO as its PCM data source. Finally, you will create some logic that integrates all these components to form a 'piano'.

Here is an overview of the entire system in \verb|ml505top| we are going to build. You may find it useful to refer to this block diagram while doing this lab.

\begin{figure}[H]
	\centerline{\includegraphics[width=\textwidth]{Lab6_Block_Diagram.png}}
\end{figure}

In this lab, you will be building the FIFOs, modifying your AC97 controller to use one, and designing the piano FSM. You will then construct a system-level testbench to verify functionality in simulation.

\subsection{Copying Files From Previous Labs}
You will need the following files from previous labs:

\begin{minted}{bash}
cd labs_sp17
cp lab4/src/uart.v lab6/src/.
cp lab4/src/uart_receiver.v lab6/src/.
cp lab4/src/uart_transmitter.v lab6/src/.
cp lab5/src/tone_generator.v lab6/src/.
cp lab5/src/synchronizer.v lab6/src.
cp lab5/src/debouncer.v lab6/src.
cp lab5/src/edge_detector.v lab6/src/.
cp lab5/src/rotary_decoder.v lab6/src/.
\end{minted}

\section{Building a Synchronous FIFO}
A FIFO (first in, first out) data buffer is a circuit that has two interfaces: a read side and a write side. The FIFO we will build in this section will have both the read and write side clocked by the same clock; this circuit is known as a synchronous FIFO.

\subsection{FIFO Functionality}
A FIFO is implemented with a circular buffer (2D reg) and two pointers: a read pointer and a write pointer. These pointers address the buffer inside the FIFO, and they indicate where the next read or write operation should be performed. When the FIFO is reset, these pointers are set to the same value. 

When a write to the FIFO is performed, the write pointer increments and the data provided to the FIFO is written to the buffer. When a read from the FIFO is performed, the read pointer increments, and the data present at the read pointer's location is sent out of the FIFO.

A comparison between the values of the read and write pointers indicate whether the FIFO is full or empty. You can choose to implement this logic as you please. The \verb|Electronics| section of the \href{https://en.wikipedia.org/wiki/FIFO_(computing_and_electronics)}{FIFO Wikipedia article} will likely aid you in creating your FIFO.

Here is a block diagram of the FIFO you should create from page 103 of the \href{https://www.xilinx.com/support/documentation/ip_documentation/fifo_generator_ug175.pdf}{Xilinx FIFO IP Manual}.

\begin{center}
	\includegraphics[height=7cm]{sync_fifo_diagram.png}
\end{center}

The interface of our FIFO will contain a subset of the signals enumerated in the diagram above.

\subsection{FIFO Interface}
Take a look at the FIFO skeleton in \verb|labs_sp17/lab6/src/fifo.v|.

Our FIFO is parameterized by these parameters:
\begin{itemize}
	\item \verb|data_width| - This parameter represents the number of bits per entry in the FIFO.
	\item \verb|fifo_depth| - This parameter represents the number of entries in the FIFO.
	\item \verb|addr_width| - This parameter is automatically filled by the log2 macro to be the number of bits for your read and write pointers.
\end{itemize}

The common FIFO signals are:
\begin{itemize}
	\item \verb|clk| - Clock used for both read and write interfaces of the FIFO.
	\item \verb|rst| - Reset synchronous to the clock; should cause the read and write pointers to be reset.
\end{itemize}

The FIFO write interface consists of three signals:
\begin{itemize}
	\item \verb|wr_en| - When this signal is high, on the rising edge of the clock, the data on \verb|din| will be written to the FIFO.
	\item \verb|[data_width-1:0] din| - The data to be written to the FIFO should be present on this net.
	\item \verb|full| - When this signal is high, it indicates that the FIFO is full.
\end{itemize}

The FIFO read interface consists of three signals:
\begin{itemize}
	\item \verb|rd_en| - When this signal is high, on the rising edge of the clock, the FIFO should present the data indexed by the write pointer on \verb|dout|
	\item \verb|[data_width-1:0] dout| - The data that was read from the FIFO after the rising edge on which \verb|rd_en| was asserted
	\item \verb|empty| - When this signal is high, it indicates that the FIFO is empty.
\end{itemize}

\subsection{FIFO Timing}
The FIFO that you design should conform to the specs above. To further, clarify here are the read and write timing diagrams from the \href{https://www.xilinx.com/support/documentation/ip_documentation/fifo_generator_ug175.pdf}{Xilinx FIFO IP Manual}. These diagrams can be found on pages 105 and 107. Your FIFO should behave similarly.

\begin{figure}[H]
	\minipage{0.48\textwidth}
	\includegraphics[width=\linewidth]{sync_fifo_write_operation.png}
	\endminipage\hfill
	\minipage{0.48\textwidth}
	\includegraphics[width=\linewidth]{sync_fifo_read_operation.png}
	\endminipage
\end{figure}

Your FIFO doesn't need to support the \verb|ALMOST_FULL|, \verb|WR_ACK|, or \verb|OVERFLOW| signals on the write interface and it doesn't need to support the \verb|VALID|, \verb|UNDERFLOW|, or \verb|ALMOST_EMPTY| signals on the read interface.

\subsection{FIFO Testing}
We have provided a testbench for your synchronous FIFO which can be found in \verb|src/fifo_testbench.v|. This testbench can test either the synchronous or the asynchronous FIFO you will create later in the project. To change which DUT is tested, comment out or reenable the defines at the top of the testbench (\verb|SYNC_FIFO_TEST|, \verb|ASYNC_FIFO_TEST|).

\begin{minted}{bash}
cd labs_sp17/lab6/sim
make CASES="tests/fifo.do"
\end{minted}

The testbench we have provided performs the following test sequence, which you should understand well.
\begin{enumerate}
	\item Checks initial conditions after reset (FIFO not full and is empty)
	\item Generates random data which will be used for testing
	\item Pushes the data into the FIFO, and checks at every step that the FIFO is no longer empty
	\item When the last piece of data has been pushed into the FIFO, it checks that the FIFO is not empty and is full
	\item Verifies that cycling the clock and trying to overflow the FIFO doesn't cause any corruption of data or corruption of the full and empty flags
	\item Reads the data from the FIFO, and checks at every step that the FIFO is no longer full
	\item When the last piece of data has been read from the FIFO, it checks that the FIFO is not full and is empty
	\item Verifies that cycling the clock and trying to underflow the FIFO doesn't cause any corruption of data or corruption of the full and empty flags
	\item Checks that the data read from the FIFO matches the data that was originally written to the FIFO
	\item Prints out test debug info
\end{enumerate}

This testbench tests one particular way of interfacing with the FIFO. Of course, it is not comprehensive, and there are conditions and access patterns it does not test. We recommend adding some more tests to this testbench to verify your FIFO performs as expected. Here are a few tests to try:

\begin{itemize}
	\item Several times in a row, write to, then read from the FIFO with no clock cycle delays. This will test the FIFO in a way that it's likely to be used when buffering user I/O.
	\item Try writing and reading from the FIFO on the same cycle. This will require you to use fork/join to run two threads in parallel. Make sure that no data gets corrupted.
\end{itemize}

\section{Asynchronous FIFOs, Multi-bit Clock Crossing}
In a previous lab, we built a single-bit synchronizer, which brought an asynchronous signal from off the FPGA (buttons, rotary encoder signals) into the system clock domain. The synchronizer consisted of two D flip-flops connected in series, clocked by the system clock. This synchronizer however only works for a single bit. To synchronize an entire bus across clock domains requires a more complex synchronization scheme.

One solution, among others, is an asynchronous FIFO which works like a synchronous FIFO, except for the fact that the read and write interfaces are clocked by different clocks (with no known phase or frequency relation).

For this lab, we want to allow communication between our AC97 controller and the piano FSM. These parts operate in different clock domains (12.288 Mhz and 33 Mhz), so we need a synchronization element to safely transfer data between them.

\subsection{Async FIFO Construction}
An asynchronous FIFO is constructed similarly to a synchronous FIFO with a two ported RAM, a read and write pointer, and some logic to generate the full and empty signals. One difference is that the two ported RAM has two independently clocked ports. Another difference is that the read and write pointers need to be properly transferred to the other clock domain before going through the full and empty signal generation logic. Here is an overview of the internals of an async FIFO from the Xilinx FIFO IP Manual, page 100.

\begin{center}
	\includegraphics[height=13cm]{async_fifo_diagram.png}
\end{center}

Notice that there is a clear divide between the two different clock domains. The two arrows that cross the clock domain are the gray-coded write and read pointers. This clock domain crossing should be performed by using 2 registers in series clocked by the other clock domain, just like the 1-bit synchronizer. This will provide a robust clock crossing to avoid data corruption.

This means that the binary write pointer should be converted to a gray-coded write pointer, before passing through 2 registers in series clocked by the read clock. Then, the gray-coded write pointer has been successfully moved to the read clock domain, where it can be converted back to binary and compared against the read pointer for generating the \verb|empty| signal. The same logic chain applies to the read pointer transfer into the write clock domain.

As your build the async FIFO, you might find it useful to create binary to gray and gray to binary modules that you can instantiate in your design. The \href{https://en.wikipedia.org/wiki/Gray_code}{gray code Wikipedia article} has some psuedocode that can help you write these modules in the 'Converting to and from Gray code' section. You can make the assumption that the async FIFO for this lab and coming project will not have read and write pointers greater than 16 bits, and can design your gray code converters appropriately.

\subsection{Async FIFO Reset Behavior}
Handling resets in an async FIFO is tricky since a synchronous reset from one clock domain won't necessarily be an appropriate reset for the other clock domain. Here are a couple reset strategies:

\begin{itemize}
	\item Use one synchronous reset signal with respect to one clock domain, and propagate the reset to the other clock domain internally with a 1-bit synchronizer.
	\item Use two reset signals, one for each clock domain, and have each reset be synchronous to its respective clock.
	\item Use one global asynchronous reset signal to reset both clock domains at once.
	\item Don't use an explicit external reset signal but define initial reg values which can be synthesized for an FPGA.
\end{itemize}

For this lab, we are going to go with the last option as it is the most robust for our target (an FPGA and a temperamental AC97 bit clock). This means that you should initialize any \verb|reg| nets that represent actual registers in your async FIFO design.

\subsection{Async FIFO Interface}
The interface of the async FIFO template in \verb|src/async_fifo.v| is identical to the interface of the synchronous FIFO with the exception of having an independent read and write clock.

\subsection{Async FIFO Timing}
The timing of the async FIFO is similar to that of the synchronous FIFO with the exception that the full and empty signals can have 'delayed' transitions (of a few clock cycles) due to the need to synchronize the read and write pointers. Refer to page 110 of the \href{https://www.xilinx.com/support/documentation/ip_documentation/fifo_generator_ug175.pdf}{Xilinx FIFO IP Manual} for the timing diagrams.

\subsection{Async FIFO Testing}
The same testbench used for testing the synchronous FIFO can be used to test the async FIFO at \verb|src/fifo_testbench.v|. The only change is that at the top of the testbench file, uncomment the \verb|ASYNC_FIFO_TEST| define and comment out the \verb|SYNC_FIFO_TEST| define. Then run the testbench as usual with \verb|make sim|.

The same additional tests recommended with the synchronous FIFO are recommended for this FIFO as well. These additional tests are even more important to get right with the asynchronous FIFO, so we highly recommend writing them.

By now, you should have a working async FIFO, at least in simulation. As you may discover, basic functional simulation isn't usually a good way of telling whether a clock crossing works as expected, although it will confirm that your logic is sound.

\section{Modifying the AC97 Controller}
We want to connect an async FIFO between the AC97 controller and the piano FSM. So, we need to modify the AC97 controller from last lab to take the \verb|dout[19:0]|, \verb|empty| signals as inputs and drive the \verb|rd_en| signal as an output. We will use the data stored in this FIFO as the PCM data we want to send for each AC97 frame.

Remove the \verb|square_wave| input from the AC97 controller's ports, and add new ports for interfacing with an async FIFO. Then, modify the logic in your AC97 controller to pull one piece of data from the async FIFO for each frame. You can decide to pull the data right after you have sent slot 4, or right before you send the tag; your choice.

If the FIFO is empty when you attempt to pull data from it, you should continue sending the last value that you had pulled from it on every AC97 frame, until the FIFO is no longer empty on the subsequent pull attempt.

\subsection{Modify the AC97 Controller Testbench}
Copy over the AC97 controller testbench (\verb|lab5/src/ac97_controller_testbench.v|) from lab 5.

Modify the AC97 controller testbench by instantiating an async FIFO and sending data to it using the \verb|system_clock| in the \verb|initial| block. Then execute the testbench again, and verify that your AC97 controller is able to properly interface with the async FIFO. You may have to modify the default parameter values when you instantiate the \verb|async_fifo|.

\section{Building the Piano FSM}
Now we will design the logic that interfaces the FIFOs coming from the UART and the async FIFO that provides data for the AC97 controller. This module is the 'Piano FSM' in the block diagram in the lab intro.

The skeleton for the \verb|piano_fsm| is provided in \verb|src/piano_fsm.v|. You can see that it has access to the UART transmitter FIFO, the UART receiver FIFO, and the AC97 sample async FIFO. It also has access to a reset signal coming from the \verb|CPU_RESET| button and the rotary decoder outputs. This FSM should implement the following functionality:

\begin{itemize}
	\item When the UART receiver FIFO contains a character, the FSM should pull the character from the FIFO and echo it back without modification through the UART transmitter FIFO.
	
	\item Once a character is pulled, its corresponding \verb|tone_switch_period| should be read from the supplied \verb|piano_scale_rom.v|.
	
	\item For a given amount of time (\verb|note_length|), the tone should be played by sending samples of the tone (at 48 kHz) into the AC97 sample FIFO
	
	\item The \verb|note_length| should default to 1/5th of a second, and can be changed by a fixed amount with the rotary wheel. This should be similar to the tempo changing you implemented in the \verb|music_streamer|.
	
	\item Through doing all of this, your FSM should take care to ensure that if a FIFO is full, that it waits until it isn't full before pushing through data.
	
	\item If the UART receiver FIFO is empty, the AC97 FIFO can be filled with a constant value (like 0) for every sample or not filled at all. The \verb|piezo_speaker| output shouldn't oscillate if there's nothing to play.
	
	\item The \verb|piezo_speaker| output of this FSM connects directly to the piezo on the ML505 board. It should be driven with the square wave that's playing through the AC97 controller.
\end{itemize}

You don't need to design the \verb|piano_fsm| as an explicit FSM with states; the design is entirely up to you.

A ROM containing mappings from ASCII character codes to the \verb|tone_switch_period| of the note to be played can be found in \verb|src/piano_scale_rom.v|. If you wish to re-generate this file, use these commands:

\begin{minted}{bash}
cd labs_sp17/lab6
python scripts/piano_scale_generator.py scale.txt
python scripts/rom_generator.py scale.txt src/piano_scale_rom.v 256 24
\end{minted}

You will then have to modify \verb|piano_scale_rom.v| to use the module name \verb|piano_scale_rom|.

A possible implementation of this module would be to include an instance of your \verb|tone_generator|, and sample its output at 48 kHz to send samples into the AC97 sample FIFO. The details of the implementation are all up to you.

It is possible that the UART receiver FIFO can fill up with samples so fast that the piano FSM can't keep up; similar overflow conditions are possible with other parts of this system. You don't need to concern yourself with detecting 'backpressure' on the entire system and can just assume that your FIFOs are large enough to buffer all the user input and audio output.

\subsection{Modify ml505top}
Now open up \verb|ml505top.v| and modify it at the bottom to include the new modules you wrote. Wire up the FIFOs and your piano FSM according to the block diagram in the lab intro. You will have to add a few lines of logic (purple cloud) representing the bridge between the ready/valid interface and the FIFO's rd\_en, wr\_en / full, empty interface. 

Make sure that you parameterize your FIFOs properly so that they have the proper \verb|data_width|. You can make your FIFOs as deep as you want, but 8 is a good default depth.

\section{Writing a System-Level Testbench}
This design involves quite a few moving parts that communicate with each other. We want to make sure that the complete integration of our system works as expected. To that end, you will have to write a system-level testbench that stimulates the top-level of your design and observes the top-level outputs to confirm correct behavior.

We have provided a template for a system testbench in \verb|src/system_testbench.v|. It will be up to you to fill in the \verb|initial| block to test all the parts of the piano.

To make the waveform shorter and easier to debug, it is suggested that you change your \verb|note_length| default value in the \verb|piano_fsm| to something much smaller than 1/5th of a second, just for testing.

\section{FPGA Testing}
To put your design on the FPGA, run \verb|make| in the \verb|labs_sp17/lab6| folder as usual. You should check the \verb|make report| to see if there are any unexpected warnings. You should watch out specifically for warnings like 'found x-bit latch' or 'x signal unconnected' or 'x signal assigned but never used'. If you see that the synthesis tool inferred a latch, you should definitely fix that warning


\section{Conclusion + Checkoff}
You are done with lab 6! Please write down any and all feedback and criticism of this lab and share it with the TA. This is a brand new lab and we welcome everyone's input so that it can be improved.


\subsection{Checkoff Tasks}
\begin{enumerate}
	\item Show the system-level testbench you wrote and its methodology for testing your piano's functionality
	\item Show the output waveform of your testbench and explain how data moves through your system
	\item Demonstrate the piano working on the FPGA both through the piezo speaker and the AC97 codec's headphone output. Demonstrate that the volume control of the AC97 controller still works.
\end{enumerate}

\end{document}